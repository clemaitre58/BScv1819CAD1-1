\documentclass[12pt]{TDTP}


\newcommand{\auteur}{C\'edric Lemaitre}
\newcommand{\couriel}{c.lemaitre58@gmail.com}
\newcommand{\promo}{Bachelor in Computer Vision}
\newcommand{\annee}{2018-2019}
\newcommand{\matiere}{Computer Aided Design 1}

\newcommand{\tdtp}{Practice}
\renewcommand{\sujet}{Intro to Matlab}


\begin{document}
\titre

\textbf{NOTE}\\
For each problem you shall create a script, for example \texttt{problem1.m}, containing all commands to answer the questions.

%%%%%%%%%%%%
\Exo
Make the following variables
\begin{enumerate}
\item $a = \begin{bmatrix} 3.14 \; 15 \; 9 \; 26 \end{bmatrix}$
\item $b = \begin{bmatrix} 2.71 \\ 7 \\ 2.1 \\ 71\\ \end{bmatrix}$
 \item $c = \begin{bmatrix} 5 \\ 4.8 \\ \vdots \\ -4.8\\ -5 \\ \end{bmatrix}$ (all the numbers from 5 to -5 in increments of -0.2).
 \item $A = \begin{bmatrix} 
 2 & \ldots & 2 \\ 
 \vdots & \ddots & \vdots\\ 
 2 & \ldots & 2 \\
 \end{bmatrix}$ a $9\times 9$ matrix full of 2's (use the commands \textbf{ones} or \textbf{zeros})
\item $B = \begin{bmatrix} 
 1 & 0 & \ldots & & 0 \\ 
 0 & \ddots & 0 \ddots & \\ 
 \vdots & 0 & 5 & 0 & \vdots \\
  & \ddots & 0 & \ddots & 0 \\
  0 &  & \ldots & 0 & 1\\
 \end{bmatrix}$ 
 a $9\times 9$ matrix of all zeros, but with the values $[1 \; 2 \; 3 \;  4 \; 5 \;  4 \;  3 \;  2 \;  1]$ on the main diagonal, use \textbf{zeros} and \textbf{diag}.
\item $C = \begin{bmatrix} 
 1 & 11 & \ldots & 91 \\ 
2 & 12 & \ldots & 92 \\ 
\vdots & \vdots & \ddots & \vdots \\ 
10 & 20 & \ldots & 100 \\ 
 \end{bmatrix}$
 a $10\times 10 $ matrix where the vector 1:100 runs down the columns (use \textbf{reshape})
 \item Create a $5\times 5$ matrix $D$ of random integers with values on the range -3 to 3. Use \textbf{rand} and \textbf{floor} or \textbf{ceil}.
\end{enumerate}

%%%%%%%%%%%%
\Exo
Solve the following equations using the variables created in \textbf{Problem 1}.
\begin{enumerate}
\item $x = \frac{1}{\sqrt{2\pi 2.5^2}} e^{-a^2/(2*2.5^2)}$
\item $y = \sqrt{(a^T)^2 + b^2}$
\item $z = \log_{10}(1/c) $, remember that $\log_{10}$ is the log base 10. So you use \textbf{log10} function. 
\end{enumerate}
Note that each of these variables is a vector of the right dimension.

%%%%%%%%%%%%
\Exo
If a matrix $A$ is defined using the MATLAB code $A=[1\; 3\; 2; \; 2\;1\;1; \; 3\; 2\; 3]$, which command will produce the following matrix
$$
B=\begin{bmatrix}
3&2\\
2&1\\
\end{bmatrix}
$$ 

%%%%%%%%%%%%
\Exo
Create the variables representing the following matrices:
$$
A=\begin{bmatrix}
1&2&3\\
2&2&2\\
-1&2&1\\
\end{bmatrix} \;\;
B=\begin{bmatrix}
1&0&0\\
1&1&0\\
1&1&1\\
\end{bmatrix} \;\;
C=\begin{bmatrix}
1&1\\
2&1\\
1&2\\
\end{bmatrix}
$$
\begin{itemize}
\item Try performing the following operations: $A+B$, $A*B$, $A+C$, $B-A$, $A*C$, $C-B$, $C*A$.
What are the results? What error messages are generated? Why?

\item What is the difference between $A*B$ and $A.*B$?
\end{itemize}

%%%%%%%%%%%%
\Exo
All points with coordinates $x=r\cos(\theta)$ and $y=\cos(\theta)$, where $r$ is a constant, lie on a circle with radius $r$. That is they satisfy the equation $x^2+y^2=r^2$.

Create a column vector for $\theta$ with the values, $0$, $\pi/4$, $\pi/2$, $3\pi/4$, and $5\pi/4$.
Take $r=2$ and compute the column vectors $x$ and $y$.

Now check that $x$ and $y$ indeed satisfy the equation of a circle, by computing the radius $r=\sqrt{(x^2+y^2)^2}$.

%%%%%%%%%%%%
\Exo
The sum of geometric series $1+r+r^2+r^3 + \cdots + r^n$, approaches the limit $\frac{1}{1-r}$ for $r<1$ as $n\rightarrow \infty$.

Take $r=0.5$ and compute the sums of series 0 to 10, 0 to 50, and 0 to 100.
Calculate the aforementioned limit and compare with your summations. Use the built-in \textbf{sum} function.

%%%%%%%%%%%%
\Exo
The number of ways to choose $k$ objects form a set of $n$ objects is defined and calculated with the formula
$$
\begin{pmatrix}
n\\k\\
\end{pmatrix}
= \dfrac{n!}{k!(n-k)!}
$$
Define a Pascal matrix with the formula 
$$
P(i,j) = \begin{pmatrix} i+j-2 \\ i-1\\ \end{pmatrix},
$$
where $i$ ranges from 1 to the number of rows and $j$ ranges from 1 to the number of columns.

Use this definition and hand calculations to find a Pascal matrix of dimension $4\times 4$.

Use Matlab's \textbf{pascal} command to check your result. 

%%%%%%%%%%%%
\Exo
Read the documention of MATLAb's \textbf{primes} command, and use it to store the first 100 primes less than or equal to 1000.
\begin{itemize}
\item Fin the sum of the first primes
\item Find the sum of the first, 20th and 97th primes.
\end{itemize}

%%%%%%%%%%%%
\Exo
Find a MATLAB one-line expression to cretae the $n\times n$ matrix $A$ satisfying
$$
a_{ij} = 
\begin{cases}
   1 & \text{if } i-j \text{ is prime} \\
   0  & \text{otherwise}
 \end{cases}
$$
%%%%%%%%%%%%
\Exo
\textbf{Manipulating variables}\\
For this problem, you need the file \emph{classGrades.mat}.
\begin{itemize}
\item Open a script and name it \emph{calculateGrades.m}. You'll write all the following command in this script.
\item Load the \emph{classGrades} file using the command \textbf{load}.
The file contains a single variable called \emph{namesAndGrades}.
\item To see how \emph{namesAndGrades} is structured, display the first 5 rows on your screen.
The first column contains the students 'names', they are just integers from 1 to 15.
The remaining 7 columns contain each student's score (on a sclae from 0 to 5) on each of 7 assignments.
There are also some NaNs which indicates that a particular student was absent on that day and didn't do the assignment.

\item We only care about the grades, so extract the submatrix containing all the rows but only columns 2 to 8 and name this new matrix \emph{grades}. To make this work for any size matrix, don't hard-code the 8, but rather use \textbf{end} or \textbf{size} commands. 
\end{itemize}

\begin{enumerate}
\item Calculate the mean score on each assignment. The result should be a 1x7 vector containing the mean grade on each assignment.

First, do this using \textbf{mean} and display the mean grades you get. What's wrong with this result?

Then use the \textbf{nanmean} command. What's different?

Name this mean vector \emph{meanGrades} (here you shoul use the vector without NaNs entries).

\item Now normalize each assignment so that the mean grade is 3.5. You'll want to divide each column of \emph{grades} by the correct element of \emph{meanGrades}.
\begin{itemize}
\item Make a matrix called \emph{meanMatrix} such that it is the same size as \emph{grades}, and each row has the values \emph{meanGrades}. 
\item Calculate the curved grades as $curvedGrades = 3.5 (grades/meanMatrix)$.
Keep in mind that you want to do the division elementwise.
\item Compute and display the mean of \emph{curvedGrades} to verify that they are all 3.5.
\item Because we divided by the mean and multiply by 3.5, it's possible that some grades that were initially close to 5 are now larger than 5.
To fix this, find all the elements in \emph{curvedGrades} that are greater than 5 and set them to 5. Use \textbf{find} command.
\end{itemize}

\item Calculate the total grade of each student and assign letter grades
\begin{itemize}
\item To calculate the \textit{totalGrade} vector, which will contain the numerical grade for each student, you want to take the mean of \textit{curvedGrades} across the columns (use \textbf{nanmean}, see help for how to specify the dimension). 
Also, we only want to end up with numbers from 1 to 5, so calculate the ceiling of the \textit{totalGrade} vector (use \textbf{ceil}).

\item Make a string called \textit{letters} that contains the letter grades in increasing order: FDCBA

\item Make the final letter grades vector \textit{letterGrades} by using \textit{totalGrade} (which should only contain values between 1 and 5) to index into \textit{letters}.

\item Finally, display the students grade using \textbf{disp}. You should find BCBBBACCBCCCCAB.
\end{itemize}
\end{enumerate}

%%%%%%%%%%%%%%%%%%%%%%%%%%%%%%%%%%%%%%%%
\end{document}
